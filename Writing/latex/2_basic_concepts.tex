\section{Basic Concepts}
	\subsection{Machine Learning}
		A sub-branch of computer science that rose to prominence and started evolving during the 1950s as part of research in the field of artificial intelligence. Machine learning refers the development of algorithms, which allow computers to learn from presented examples. The computer is thereafter supposed to learn from its collected experience and automate the process of solving similar tasks. This process is referred to by term \textit{training}.
		\\
		One official definition as coined by Tom M. Mitchell \cite{mitchell} is  "A computer program is said to learn from experience $E$ with respect to some class of tasks $ T $ and performance measure $ P $ if its performance at tasks in $ T $, as measured by $ P $, improves with experience $ E $."
		\\
		The most common use of machine learning algorithms is the analysis of real-world data for certain tasks, when a concrete programmer-written application would be ineffective in solving. Such is the case for example with problems, which a human would be able to solve, but would not be able to determine the rules for solving explicitly. Or alternatively, where the rules are not constant, but rather evolving as time progresses. The purpose of teaching a machine to solve such tasks, is modelling, prediction or detection of details or certainties about the real world. 
		\\
		Vivid examples of real world uses are speech recognition as used in cellphones or in 
		call routing system as well as visual recognition, where and algorithm is trained to recognize graphic patterns and used in medicine or in hand written text recognition.
	\subsection{Text Mining}
		{\color{red} \Large placeholder}
	\subsection{Classification}
		{\color{red} \Large placeholder}
	\subsection{Information Quality}
		
